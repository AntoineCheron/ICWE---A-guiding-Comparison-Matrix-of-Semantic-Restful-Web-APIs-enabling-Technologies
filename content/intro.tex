\section{Introduction}

\vspace*{-0.2cm}


Today, RESTful APIs \cite{FieldingThesis} have become the de-facto standard for building web applications. The main reason behind this popularity lies in the appropriate trade-off between the facility to build such applications and the benefits provided by this approach in such an opened large-scale distributed system: evolutivity, scalability and loose-coupling. However, 95\% of APIs are not RESTful ~\cite{10.1007/978-3-319-38791-8_2} as they claim.

Until today, no single standard has emerged to design truly RESTful APIs. Consequently, software architects are facing the challenge of selecting the right technologies for the design and implementation of these systems. Typically, a software architect has to select the right interface description language (IDL), interchange format and framework to ease the development of such APIs.

In addition, a new trend has recently emerged to create RESTful APIs that carry their own semantics, they are called Semantic RESTful APIs~\cite{7195633}. It is a vision that proposes to make fully REST-compliant APIs compatible with the Semantic Web ~\cite{TheSemanticWeb} and Linked Data~\cite{LinkedDataPrinciples}. From our experience at FABERNOVEL, we found that building such APIs does not require much more effort than truly RESTful systems, whereas it offers great benefits, such as loose-coupling, automated API mash-ups~\cite{benslimane2008services}, machine-interpretability and very powerful querying. 

% TODO: add a word on the questionnaire sent to developers to gather their REX on choosing hypermedia and Linked Data technologies
However, the design of semantic RESTful APIs considerably increases the complexity for the architect to choose the appropriate technology. Indeed, the specific criteria and properties to be taken into account are not explicit when choosing an IDL, an interchange format and a framework.
The industrial needs are growing for proper tools to support trade-off decisions of the architect; a tool that would help him/her  to understand the consequences of a design decision, i.e. the characteristics and limitations of each approach. 
%A decision matrix is missing that would allow the architect to understand the consequences of a design decision, i.e. the characteristics and limitations of each approach.

In this paper, we propose to fill this gap by providing three decision matrices that help architects to choose the technologies that will best fit their needs. The main contributions of this paper are:

\begin{itemize}
    \item three comparison matrices of interchange formats, interface description languages and frameworks that help choosing the appropriate set of technologies to build Semantic RESTful APIs;
    \item key features that are missing from state-of-the-art technologies to assist and make the creation of Semantic RESTful APIs more beneficial.
\end{itemize}

Using these comparison matrices, we illustrate their usage on an industrial case and draw the outline of a research road-map to ease the adoption of Semantic RESTful APIs in the industry.

The remainder of this paper is organized as follows. Section \ref{sec:background} provides the required background on Semantic REST APIs and the reference maturity model to choose the functionality level of an API along, with its limitation. The two following sections describe our comparison matrices and an illustration that highlights the benefits of our proposition. Finally, section \ref{sec:discussion} discusses the role of the existing frameworks to build Semantic REST APIs. 
