\section{Discussion} \label{sec:discussion}

\vspace*{-0.2cm}

% TODO - ne parler de GraphQL que dans related work

% Sert à rien
%As discussed in section \ref{sec:matrix} there is no framework yet available to build a Semantic REST API.
This section discusses, from our perspective, why there is no standard solution to meet all the criteria discussed in section \ref{sec:matrix}, and highlight the opportunity for new research initiatives. 

% Bien, une ou deux lignes à enlever
First, there is no standard IETF or W3C interchange format for building semantic REST APIs. In addition, none of the existing interchange formats support all the criteria described above, making it likely that new formats will emerge.
For this reason, frameworks supporting semantic REST APIs will have to rely on formats that are likely to evolve, which will require additional effort and costs. This reduces the likelihood that framework editors will invest time in developing such features. 
%And yet, this is necessary to compensate for the complexity of these technologies.

For us, the second and also the most important reason is that the well-known and widely used tools do not rely on REST semantic APIs to provide additional and useful features. Among the possible functionalities, we envision various tools to automate APIs testing, REST clients generation, API gateways and middleware and smarter desktop REST clients.
We believe that this limits the adoption of REST semantic APIs because the cost of building these APIs is not perceived as offering a sufficient short-term return on investment.


% Second, there is currently no tool that leverages the power of REST semantic APIs. Thus, being compatible with Semantic REST requires additional efforts to adapt client libraries and to train developers. We believe that with new tools, such as front-end clients, automated test libraries, improved documentation generation and HTTP clients such as Postman, developers would be more interested in semantic REST features.

