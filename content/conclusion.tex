\section{Conclusion}

\vspace*{-0.2cm}

In this paper, we have presented three comparison matrices that assist architects in choosing Semantic REST APIs enabling technologies that meet their needs.
Through a real example, we have illustrated how the use of these matrices simplifies the choice of these technologies. 
As stated in the paper, technologies should be chosen not only according to the number of criteria they meet, but also according to the specific needs of the project. 
To facilitate this selection, we have developed an assistant available online.

We also pointed out some interesting features missing in current technologies.
The description of constraints and conditions indicating the availability of state transitions is ignored by IDLs, vocabularies, interchange formats and frameworks. On the other hand, resource modeling as FSM is not available in most frameworks.
More importantly, well-known tools do not take advantage of the power of Semantic REST APIs to provide additional and useful features.

Based on these findings, we identify areas for improvement in the tools around Semantic REST APIs that we believe can increase its adoption. By leveraging the semantic description and advertising of state transitions and non-functional properties, automated testing tools can become smarter, REST client libraries can lower the coupling with the server and automate tasks such as login, and middleware can automatically create responses from the composition of several APIs.