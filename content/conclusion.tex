\section{Conclusion}

% Not to forget in this conclusion, from introduction: Based on these comparison matrices, we draw the outline of a research road-map to ease the adoption of Semantic RESTful APIs in industry.

In this paper we presented three comparison matrices that help architects choose the Semantic RESTful APIs enabling technologies that meet their needs.
We used an example to show how simple it is to reduce the list of technologies to three when the characteristics to be implemented are identified.
In future work, we plan to make these matrices available online through semantic restful APIs and to improve their accessibility to facilitate technology selection.

We also pointed out that some important features are missing in current technologies. The modeling and description of constraints and conditions indicating the availability of state transitions are ignored by IDL, vocabularies, interchange formats and frameworks. On the other hand, modeling resource as FSM is not available in most frameworks. In addition, IDLs target the documentation of the entire API in a single file. This prevents them from being used for hypermedia controls. Finally, very few tools take advantage of the power of RESTful semantic APIs, and interest around this topic is still low.

Based on these findings, we identify three areas for improvement: (i) adoption, (ii) semantics and documentation, and (iii) implementation. Adoption can be made easier and more beneficial with new tools that leverage the power of these APIs. The discovery and selection of semantic vocabulary should be simplified and the addition of new vocabulary would allow systems to be described more accurately and more thoroughly. This would reduce coupling and increase both composability and navigation.

